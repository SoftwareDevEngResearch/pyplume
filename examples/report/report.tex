%
\pdfoutput=1
\pdfminorversion=5 % necessary for EES
%
%\documentclass[12pt,number,sort&compress,preprint,review]{elsarticle}
\documentclass[smallextended,referee]{svjour3}

\smartqed

\journalname{}
% better printing of numbers
\usepackage[utf8]{inputenc}
\usepackage[T1]{fontenc}
\usepackage[english]{babel}
\usepackage{textcomp}
\usepackage{booktabs,multicol}
\usepackage[hyphens]{url}
\usepackage[breaklinks=true, linkcolor=blue, citecolor=blue, colorlinks=true]{hyperref}
\usepackage{latexsym, amsmath, amssymb}
\usepackage{graphicx, wrapfig, subcaption}
\usepackage{color}
\usepackage[pdftex,usenames,dvipsnames,changebar]{xcolor}
\usepackage[xcolor]{changebar}
\usepackage{listings}
\usepackage[shortlabels]{enumitem}
\usepackage[binary-units]{siunitx}



\definecolor{dkgreen}{rgb}{0,0.6,0}
\definecolor{gray}{rgb}{0.5,0.5,0.5}
\definecolor{mauve}{rgb}{0.58,0,0.82}

\lstset{ %
  language=C,
  breaklines=true,
  columns=flexible,
  basicstyle=\tiny \ttfamily,
  backgroundcolor=\color{white},
  showspaces=false,
  showstringspaces=false,
  showtabs=false,
  frame=single,
  tabsize=2,
  captionpos=b,
  keywordstyle=\color{blue},
  commentstyle=\color{dkgreen},
}

\graphicspath{ {examples/report/figures/} }


\begin{document}


\title{Modeling atmospheric chemistry using Cantera reactor networks.}

\titlerunning{Modeling atmospheric chemistry using Cantera reactor networks.}

\author{Anthony S.~Walker}

\institute{A.~Walker \at
           School of Mechanical, Industrial, and Manufacturing Engineering, Oregon State University, Corvallis, OR, USA \\
           \email{kyle.niemeyer@oregonstate.edu}
}

\date{Received: date / Accepted: date}

\maketitle 

% We present recent work on the swept rule, a domain decomposition strategy for explicit solutions to partial differential equations, on heterogeneous, distributed architectures, in particular those commonly found on compute clusters containing GPGPU accelerators.
\begin{abstract}
\texttt{PyPlume} is a software developed for the purpose of atmospheric modeling. The software focuses on building complicated reactor networks to more practically simulate atmospheric chemistry. 
\end{abstract}


\section{Introduction}

\begin{acknowledgements}

\end{acknowledgements}

\appendix


% \bibliographystyle{plainurl}
% \bibliography{paper-heterogeneous-swept}

\end{document}
